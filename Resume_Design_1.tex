%-------------------------------------------------------------------------------------
%-------------Updated for Course Classroom Examples and Job Search--------------------
%-----------------------------------March 2020----------------------------------------
%-----------------------------------English-Japanese Version In Development-----------
%----------Links to Updated Sample Articles Available in March 2020 Week 3 and 4------
%----------for the publication pipeline and curated review----------------------------
%----------Additional Information: Please submit the form on the Math Learning Space 
%--------------------------------Research Portfoilo ----------------------------------
%-------------------------------------------------------------------------------------
%\documentclass{TMLSStyleGuideResumeVitae}
%--------Style document from the Mathematical Learning Space style guide--------------
%--------Other Style documents can be used as well similar to journal articles/posters---
\documentclass[preprint, 8pt]{elsarticle}
\usepackage{xcolor}
\usepackage{chemfig}
\usepackage{tikz}
\usepackage{graphicx}
\usepackage{amsmath, amssymb}
\setlength{\parindent}{0pt}
\usepackage{pgfplots}
\pgfplotsset{compat=1.3,}
\pgfplotscreateplotcyclelist{line styles}{ 
	black,solid\\
	blue,dashed\\
	red,dotted\\
	orange,dashdotted\\
}
\newcommand*\GnuplotDefs{
	set samples 50;
	cdfn(x,mu,sd) = 0.5 * ( 1 + erf( (x-mu)/sd/sqrt(2)) );
	pdfn(x,mu,sd) = 1/(sd*sqrt(2*pi)) * exp( -(x-mu)^2 / (2*sd^2) );
	tpdfn(x,mu,sd,a,b) = pdfn(x,mu,sd) / ( cdfn(b,mu,sd) - cdfn(a,mu,sd) );
}
\usepackage[a4paper,bindingoffset=0.2in,
left=0.5in,right=0.5in,top=1in,bottom=1in,
footskip=.25in]{geometry}
%\usepackage{geometry}
\usepackage{mathtools}
\usepackage{tkz-berge}
\usetikzlibrary{calc} 
\usetikzlibrary{automata}
\usetikzlibrary{arrows}
\usetikzlibrary{positioning,shapes,shadows,arrows}
\usetikzlibrary{shapes.geometric}
\usetikzlibrary{calendar,shadings}
\renewcommand*{\familydefault}{\sfdefault}
\colorlet{winter}{blue}
\colorlet{spring}{green!60!black}
\colorlet{summer}{orange}
\colorlet{fall}{red}
\newcount\mycount
\newcommand\shapeLarge{50mm}
\newcommand\shapeMedium{25mm}
\newcommand\shapeSmall{5mm}
\newcommand*{\xMin}{0}
\newcommand*{\xMax}{6}
\newcommand*{\yMin}{0}
\newcommand*{\yMax}{6}
\newcommand*{\zMax}{6}
\newcommand*{\zMin}{0}
\definecolor{colorwaveA}{RGB}{98,145,224}
\definecolor{colorwaveB}{RGB}{250,250,50}
\definecolor{colorwaveC}{RGB}{25,125,25}
\definecolor{colorwaveD}{RGB}{100,100,100}
\definecolor{colorwaveE}{RGB}{80,100,1}
\definecolor{colorwaveF}{RGB}{60,1,1}
\definecolor{colorwaveG}{RGB}{25,1,100}
\definecolor{colorwaveH}{RGB}{1,90,1}
\definecolor{colorwaveI}{RGB}{1,100,1}
\definecolor{colorwaveJ}{RGB}{1,1,1}
\tikzset{
	shapeTriangle/.style={draw,shape=regular polygon,fill=colorwaveA,circular drop shadow,regular polygon sides=3,minimum size=\shapeSmall,inner sep=0pt,outer sep=0pt},
	shapeTriangle3/.style={shapeTriangle,fill=colorwaveD,circular drop shadow,shape border rotate=45},
	shapeTriangle4/.style={shapeTriangle,fill=colorwaveA,circular drop shadow,shape border rotate=90},
	shapeTriangle5/.style={shapeTriangle,fill=colorwaveB,shape border rotate=135},
	shapeTriangle6/.style={shapeTriangle,fill=colorwaveC,shape border rotate=180},
	shapeTriangle7/.style={shapeTriangle,fill=colorwaveE,shape border rotate=225},
	shapeTriangle8/.style={shapeTriangle,fill=colorwaveF,shape border rotate=270},
	shapeTriangle9/.style={shapeTriangle,fill=colorwaveG,shape border rotate=315},
}

\tikzset{
	shapeSquare/.style={draw,shape=regular polygon,fill=colorwaveC,circular drop shadow,regular polygon sides=4,minimum size=\shapeSmall,inner sep=0pt,outer sep=0pt},
	shapeSquare2/.style={shapeSquare,shape border rotate=45},
}

\tikzset{
	shapeHexagon/.style={draw,shape=regular polygon,fill=colorwaveA,circular drop shadow,regular polygon sides=6,minimum size=\shapeSmall,inner sep=0pt,outer sep=0pt},
	shapeHexagon2/.style={shapeHexagon,shape border rotate=90},
}

\tikzset{
	shapeOctagon/.style={draw,shape=regular polygon,fill=colorwaveB,circular drop shadow,regular polygon sides=8,minimum size=\shapeSmall,inner sep=0pt,outer sep=0pt},
	shapeOctagon2/.style={shapeHexagon,shape border rotate=45},
}
\tikzset{
	shapeEllipse/.style={draw,shape=ellipse,minimum size=\shapeSmall,inner sep=0pt,outer sep=0pt},
	shapeEllipse2/.style={shapeEllipse,shape border rotate=90},
}

\tikzset{
	closedFigure/.style={draw=\draw[->,rounded corners=0.2cm,shorten >=2pt]
		(1.5,0.5)-- ++(0,-1)-- ++(1,0)-- ++(0,2)-- ++(-1,0)-- ++(0,2)-- ++(1,0)--
		++(0,1)-- ++(-1,0)-- ++(0,-1)-- ++(-2,0)-- ++(0,3)-- ++(2,0)-- ++(0,-1)--
		++(1,0)-- ++(0,1)-- ++(1,0)-- ++(0,-1)-- ++(1,0)-- ++(0,-3)-- ++(-2,0)--
		++(1,0)-- ++(0,-3)-- ++(1,0)-- ++(0,-1)-- ++(-6,0)-- ++(0,3)-- ++(2,0)--
		++(0,-1)-- ++(1,0)}
}
\tikzstyle{start}=[circle, draw=none,,minimum size=\shapeMedium, fill=blue, circular drop shadow,text centered, anchor=north, text=white]
\tikzstyle{finish}=[circle, draw=none,,minimum size=\shapeMedium, fill=blue,circular drop shadow,text centered, anchor=north, text=white]
\tikzstyle{finish}=[rectangle, draw=none, ,minimum size=\shapeMedium,fill=blue,circular drop shadow,text centered, anchor=north, text=white]
\usepackage[noadjust]{cite}
\usepackage{algpseudocode}
\usepackage{listings}
\usepackage{algorithm}
\usepackage{color}
\usepackage{parskip}
\usepackage{amsfonts}
\usepackage{amsthm}
\usepackage{tikz}
\usepackage{tkz-berge}
\usepackage{caption}
\usepackage{hyperref}
\usepackage{amsrefs}
\usepackage{mathtools, amssymb}
\usepackage{graphicx}
\usepackage{subcaption}
\usepackage{tabularx,ragged2e}
\usepackage[framemethod=tikz]{mdframed}
\newcommand{\N}{\mathbb N}
\newcommand{\Q}{\mathbb Q}
\theoremstyle{definition}
\newtheorem{definition}{Definition}[section] 
\newtheorem{theorem}{Theorem}[section]
\newtheorem{example}{Example}[section]
\renewcommand{\qedsymbol}{$\blacksquare$}
\newtheorem{corollary}{Corollary}[theorem]
\newtheorem{lemma}[theorem]{Lemma}
\renewcommand{\rmdefault}{ptm} 
\definecolor{c1}{rgb}{0.858, 0.188, 0.478}
\definecolor{c2}{RGB}{219, 48, 122}
\definecolor{c3}{cmyk}{0, 0.7808, 0.4429, 0.1412}
\definecolor{c4}{gray}{0.1}
\definecolor{c5}{RGB}{142, 68, 173}
\definecolor{blueish}{rgb}{0.565,0.886,1} 
\definecolor{greenish}{rgb}{0.565,1,0.886}
\definecolor{darkgray}{rgb}{0.15,0.15,0.15} 
\definecolor{lightgray}{rgb}{0.6,0.6,0.6}
\definecolor{shizen}{RGB}{242,246,246}
\usepackage{CJKutf8} 
\newcounter{dummy}
\graphicspath{{Figures/}}
\twocolumn
\begin{document}
%----------------------------------------------------------------------------%
%----------------------------------------Contact Information-----------------%
%----------------------------------------------------------------------------%
\begin{frontmatter}
\title{Jeff Cromwell, PhD} 
\end{frontmatter}
\centerline{\textcolor{c4}{Technical Resume}}
\begin{CJK}{UTF8}{min}	
%-----------------------------------------------------------------------------------%
%-------------------Selected Research and Teaching Positions------------------------%
%-----------------------------------------------------------------------------------%
\section{\centerline{\textcolor{c4}{\tiny University Research and Teaching Positions}}}
\begin{table}[H]
\tiny
\centering
\begin{tabular}{p{0.5cm}p{2.5cm}p{2cm}p{2cm}}
 \hline
 Year & University/Company & Location & Title \\ 
 \hline
2017 & Arizona State University -College of Health Solutions-Department of 
Biomedical Informatics & Tempe Arizona & Web Application Developer \\
2016 & Wyzant & Tempe Arizona & Mathematics Tutor \\
2010-2011 & University of Pittsburgh - School of Medicine- Department of 
Biomedical Informatics & Pittsburgh Pa & Lecturer Research Statistical Software Developer \\
 \hline
\end{tabular}
\end{table}
%-------------------------------------------------------------------------------------------%
%----------------------------Mathematical Learning Space-----------------------------%
%-------------------------------------------------------------------------------------------%
\begin{enumerate}\tiny
\item {Weebly} {\href{http://mathlearningspace.weebly.com/}{\textcolor{c5}{Math Learning Space-Weebly}}}
\item {GitHib} {\href{https://github.com/MathematicalLearningSpace$}{ \textcolor{c5}{Mathematical Learning Space-GitHub}}}
\item {TMLSRP} {\href{http://mathlearningspace.weebly.com/}{Course 1: Differential Equations in 
Mathematical Biology, Botany, Chemistry and Oncology}}
\item {TMLSRP} {\href{http://mathlearningspace.weebly.com/}{Course 2: Design of Signal Transduction 
Networks of Molecular Interaction in Oncology Cell Lines}}
\item {TMLSRP} {\href{http://mathlearningspace.weebly.com/}{Course 3: Molecular Machine Learning 
with Topological Dynamics for Compound Discovery in Organelles}}
\item {TMLSRP} {\href{http://mathlearningspace.weebly.com/}{Course 4: Music and Mathematics Classical Music 
and Smooth Jazz Modeling, Analysis and Simulation}}
\end{enumerate}
%-------------------------------------------------------------------------------------------%
%----------------------------------------Education Section-------------------------------%
%-------------------------------------------------------------------------------------------%
\section{\centerline{\textcolor{c4}{\tiny Education}}}\tiny
\begin{enumerate}
\item{2004}{ Ph.D. Natural Resource Economics- \textcolor{c5}{Emphasis: Nonlinear Time Series Analysis 
and Chaos Theory}}{West Virginia University} 
\item{1998} { M.S. Agricultural Economics- \textcolor{c5}{Emphasis: Mathematical Statistics} }{West Virginia University}
\item{1986} { B.A. Economics- \textcolor{c5}{Emphasis: Mathematics and Philosophy} }{California University of PA}
\end{enumerate}
%-------------------------------------------------------------------------------------------%
%------------------------Lecture Posters and Programs Section------------------------%
%-------------------------------------------------------------------------------------------%
\section{\centerline{\textcolor{c4}{\tiny Course Lectures and Articles}}}
In this section based on course 1 in the 
Mathematical Learning Space these examples are 
representations of articles from the courses.  
Here the 3500 words 3-5 page 2 column data structure in XML, HTML and 
TeX with 3-5 tables and a 1-3 3 x 3 matrix of figures is the 
design pattern for scientific communication in this field. 
Some have more than 3 tables and 9 figures.
\begin{table}[ht]\tiny
\centering
\begin{tabular}{p{1cm}p{6cm}}
 \hline
 ID & Journal Name \\ 
 \hline
CMA & {\href{http://mathlearningspace.weebly.com/}{\textcolor{c5}{Computer and Mathematics with Applications}}} \\
AM & {\href{http://mathlearningspace.weebly.com/}{\textcolor{c2}{Discrete Applied Mathematics}}}  \\
EM & {\href{http://mathlearningspace.weebly.com/}{\textcolor{c1}{Economic Modelling}}} \\
AML & {\href{http://mathlearningspace.weebly.com/}{\textcolor{c2}{Applied Mathematics Letters}}} \\
SPL & {\href{http://mathlearningspace.weebly.com/}{\textcolor{c3}{Statistics and Probability Letters}}} \\
NN & {\href{http://mathlearningspace.weebly.com/}{\textcolor{c4}{Neural Networks}}} \\
CSDA & {\href{http://mathlearningspace.weebly.com/}{\textcolor{c1}{Computational Statistics and Data Analysis}}} \\
IJMI & {\href{http://mathlearningspace.weebly.com/}{\textcolor{c5}{International Journal of Medical Informatics}}} \\
AIMS & {\href{http://mathlearningspace.weebly.com/}{\textcolor{c5}{AIM Mathematics Journal}}} \\
CLJ & {\href{http://mathlearningspace.weebly.com/}{\textcolor{c5}{Cancer Letter Journals}}} \\
MTS & {\href{http://mathlearningspace.weebly.com/}{\textcolor{c5}{Music Theory Spectrum}}} \\
IJC & {\href{http://mathlearningspace.weebly.com/}{\textcolor{c5}{Indian Journal of Cancer}}} \\
 \hline
\end{tabular}
\end{table}
%--------------------------------------------------------------------------------------%
%-----------------------------Course 1-----------------------------------------------%
%--------------------------------------------------------------------------------------%
\section{\centerline{\textcolor{c4}{\tiny Course 1: Differential Equations in Mathematical Biology, Botany, Chemistry and Oncology I}}}
\begin{enumerate}\tiny
\item Course 1 Lecture 1-3 Article: \href{}{Sample: DNA Shape Categories for Statistical Learning Models of Gene Overexpression, Amplification, Mutation and Regulation in GastroIntestinal Cancer 胃腸癌における遺伝子過剰発現、増幅、突然変異および調節の統計学習モデルのためのDNA形状カテゴリー Ichō gan ni okeru idenshi kajō hatsugen, zōfuku, totsuzenhen'i oyobi chōsetsu no tōkei gakushū moderu no tame no dīenuē keijō kategorī}
\item Course 1  Lecture 4-6 Article: \href{}{Sample: Delayed Differential Equations for Intestinal Metaplasia of Gene Expression Modulation in Gastric Cancer 胃癌における遺伝子発現調節の腸化生のための遅延微分方程式 Igan ni okeru idenshi hatsugen chōsetsu no chō-ka-sei no tame no chien bibun hōteishiki}
\item Course 1  Lecture 7-9 Article: \href{}{Sample: Mucomodulators and Mucin Dependent Oncogenic cell Signaling and Immunomodulation in Gastric Cancer 胃癌におけるムコモジュレーターとムチン依存性発癌性細胞シグナル伝達および免疫調節 Igan ni okeru mukomojurētā to muchin isonsei hatsugan-sei saibō shigunaru dentatsu oyobi men'eki chōsetsu}
\item Course 1  Lecture 10-12 Article: \href{}{Sample: A Mathematical Model of Molecular Complexity and  Epigenetic Modifications with Mucin Regulation in GastroIntestinal Cancers 胃腸癌におけるムチン調節を伴う分子複雑性および後成的修飾の数学モデル Ichō gan ni okeru muchin chōsetsu o tomonau bunshi fukuzatsu-seioyobi go Sei-teki shūshoku no sūgaku moderu}
\item Course 1  Lecture 13-15 Article: \href{}{Sample: Minimal Spanning Trees and Multivariate Nonparametric Distributional Testing for Gastric Cancer Chemosensitivity 胃癌化学感受性のための最小全域木と多変量ノンパラメトリック分布試験 Igan kagaku kanjusei no tame no saishō zen'ikigi to ta henryō nonparametorikku bunpu shiken}
\item Course 1  Lecture 16-18 Article: \href{}{Sample: A Mathematical Model of Multi-functional Regulators of Gut Homeostasis with Microbiota Diversity 腸内恒常性の微生物叢多様性の多機能レギュレーターの数学的モデル Chōnai kōjō-sei no biseibutsu kusamura tayō-sei no ta kinō regyurētā no sūgaku-teki moderu}
\item Course 1  Lecture 19-21 Article: \href{}{Sample:Differential Equation Specification  and Polysyllabic Filtering for Medical and Chemical Vocabulary Interaction Models in GastroIntestinal Cancer Research 胃腸癌研究における医学的および化学的語彙相互作用モデルのための微分方程式仕様と多音節フィルタリング Ichō gan kenkyū ni okeru igaku-teki oyobi kagaku-teki goi sōgo sayō moderu no tame no bibun hōteishiki shiyō to taonsetsu firutaringu}
\item Course 1  Lecture 22-24 Article: \href{}{Sample: Stability Classification Designs for Differential Equation Systems of DNA, Protein and Compound Interaction Combinatorics DNA、タンパク質、化合物相互作用の組み合わせ方程式の微分方程式システムの安定性分類設計 Dīenuē, tanpakushitsu, kagōbutsu sōgo sayō no kumiawase hōteishiki no bibun hōteishiki shisutemu no antei-sei bunrui sekkei}
\item Course 1  Lecture 25-27 Article: \href{}{Sample: A Delayed Differential Equation Model of Phytochemicals in GastroIntestinal Cancer 胃腸癌における植物化学物質の遅延微分方程式モデル Ichō gan ni okeru shokubutsu kagaku busshitsu no chien bibun hōteishiki moderu}
\item Course 1  Lecture 28-30 Article: \href{}{Sample: Delayed Fractional Differential Equation Models in Gastrointestinal Cancer 消化器癌における遅延分数微分方程式モデルShōkakigan ni okeru chien bunsū bibun hōteishiki moderu}
\end{enumerate}
%--------------------------------------------------------------------------------------%
%------------------------------Course 1----------------------------------------------%
%--------------------------------------------------------------------------------------%
\section{\centerline{\textcolor{c4}{\tiny Course 1: Differential Equations in Mathematical Biology, Botany, Chemistry and Oncology II}}}
Current Research Work in Viral Dynamics based on the work in Course 1 and 2.
\begin{enumerate}
\item  Course 1  Lecture 19-21 Article: \href{}{Sample: A Mathematical Model of Viral Dynamics, Mutations and the Immune System  ウイルスのダイナミクス、突然変異、免疫系の数学モデル Uirusu no dainamikusu, totsuzenhen'i, men'eki-kei no sūgaku moderu}
\item  Course 1  Lecture 19-21 Article: \href{}{Sample: Nonlinear Time Series and Marginal Regression Models for International Viral Dynamics in Epidemiological Surveillance Networks  疫学的監視ネットワークにおける国際ウイルス動力学のための非線形時系列および周辺回帰モデル Ekigaku-teki kanshi nettowāku ni okeru kokusai uirusu dōryokugaku no tame no hisenkei jikeiretsu oyobi shūhen kaiki moderu}
\item  Course 1  Lecture 19-21 Article: \href{}{Sample: Threshold Identification and Classification in a Nonlinear Time Series Analysis of International Viral Dynamics in EARS Epidemiological Surveillance Networks }
\item  Course 1  Lecture 19-21 Article: \href{}{Sample: A Mathematical Model of Immunopathogenesis of Rheumatoid Arthritis with effects from Tocilizumab Dose Efficacy Treatment  サンプル:トシリズマブの用量有効性治療の効果を伴う関節リウマチの免疫病原性の数学モデル Sanpuru: Toshirizumabu no yōryō yūkōsei chiryō no kōka o tomonau kansetsu riumachi no men'eki byōgen-sei no sūgaku moderu}
\item  Course 1  Lecture 19-21 Article: \href{}{Sample: A Mathematical Model for Chloroquine Phagosome Interaction  サンプル:クロロキンファゴソームの相互作用の数学モデル Sanpuru: Kurorokinfagosōmu no sōgo sayō no sūgaku moderu}
\item  Course 1  Lecture 19-21 Article: \href{}{Sample: A Mathematical Model of Respiratory Disease and Pulmonary Kinetic Processes  サンプル:呼吸器疾患と肺の運動過程の数学モデル Sanpuru: Kokyūkishikkan to hai no undō katei no sūgaku moderu}
\item  Course 1  Lecture 19-21 Article: \href{}{Sample: A Mathematical Model of VP1-4 Spherical Viral Capsid Disassembly  サンプル:VP1-4球状ウイルスカプシド分解の数学モデル Sanpuru: VP 1 - 4 kyūjō uirusukapushido bunkai no sūgaku moderu}
\item Course 1  Lecture 19-21 Article: \href{}{Sample:  A Mathematical Model of Stress Induced Upregulation in Protein Responses  ストレスの数学的モデルは、タンパク質応答のアップレギュレーションを誘発しました Sutoresu no sūgaku-teki moderu wa, tanpakushitsu ōtō no appuregyurēshon o yūhatsu shimashita}
\item Course 1  Lecture 22-24 Article: \href{}{Sample: A Mathematical Model of the Assembly of a 20S Proteasome with Peptide Post-Translational Modification ペプチド翻訳後修飾を伴う20Sプロテアソームのアセンブリの数学モデル Pepuchido hon'yakugoshūshoku o tomonau 20 esupuro teasōmu no asenburi no sūgaku moderu}
\item Course 1  Lecture 25-27 Article: \href{}{Sample: A Molecular Machine Learning Algorithm of Multi-Protein Complexes in The Digestion System  消化システムにおけるマルチタンパク質複合体の分子機械学習アルゴリズム Shōka shisutemu ni okeru maruchi tanpakushitsu fukugō-tai no bunshi kikai gakushū arugorizumu}
\item Course 1  Lecture 28-30 Article: \href{}{Sample: A Mathematical Model of Normal Mode Superposition in Compound-Protein Interaction 化合物間相互作用におけるノーマルモード重ね合わせの数学モデル Kagōbutsu-kan sōgo sayō ni okeru nōmarumōdo kasane-awase no sūgaku moderu}
\end{enumerate}
%--------------------------------------------------------------------------------------%
%-----------------------------Course 2-----------------------------------------------%
%--------------------------------------------------------------------------------------%
\section{\centerline{\textcolor{c4}{\tiny Course 2: Design of Signal Transduction Networks of Molecular Interaction in Oncology Cell Lines}}}
\begin{enumerate}
\item Course 2 Lecture 1-3 Article: \href{}{Sample:A Mathematical model of Cell Cycle Process Regulation of Cell Cycle, Chromosome Segregation and G2/M Transition 細胞周期、染色体分離およびG2 / M遷移の細胞周期プロセス調節の数学モデル Saibō shūki, senshokutai bunri oyobi G 2/ M sen'i no saibō shūki purosesu chōsetsu no sūgaku moderu}
\item Course 2  Lecture 4-6 Article: \href{}{ Sample: A Mathematical Model of Teleomere Maintenance and Telomeric DNA Binding  テロメアの維持とテロメアDNA結合の数学モデル Teromea no iji to teromea dīenuē ketsugō no sūgaku moderu}
\item Course 2 Lecture 7-9 Article: \href{}{Sample: A Mathematical Model of Mitotic Biological Processes in a Gene Ontology 遺伝子オントロジーにおける有糸分裂生物学的プロセスの数学モデル Idenshi ontorojī ni okeru yūshibunretsu ikimonogaku-teki purosesu no sūgaku moderu}
\item Course 2 Lecture 10-12 Article: \href{}{Sample: A Mathematical Model of DNA Repair Genes/Proteins Based on Molecular Function and Signaling Pathway 分子機能とシグナル伝達経路に基づくDNA修復遺伝子/タンパク質の数学モデル Bunshi kinō to shigunaru dentatsu keiro ni motodzuku dīenuē shūfuku idenshi/ tanpakushitsu no sūgaku moderu}
\item Course 2 Lecture 13-15 Article: \href{}{Sample:A Mathematical Model of Glycoproteins of the Epithelia Mucosa in the Mucocilary System 粘液系の上皮粘膜の糖タンパク質の数学モデル Nen'eki-kei no jōhi nenmaku no tō tanpakushitsu no sūgaku moderu}
\item Course 2 Lecture 16-18 Article: \href{}{Sample: A Mathematical Model of Transcriptional Activity for Helix-Loop-Helix Proteins ヘリックスループヘリックスタンパク質の転写活性の数学モデル Herikkusurūpuherikkusutanpaku-shitsu no tensha kassei no sūgaku moderu}
\item Course 2 Lecture 19-21 Article: \href{}{Sample: A Delay Differential Equation Model for Signal Transduction in Hepatocellular Carcinoma 肝細胞癌におけるシグナル伝達の遅延微分方程式モデル Kansaibōgan ni okeru shigunaru dentatsu no chien bibun hōteishiki moderu}
\item Course 2 Lecture 22-24 Article: \href{}{Sample: A Mathematical Model of Heterodimer DNA helix Bending  ヘテロダイマーDNAヘリックス曲げの数学モデル Heterodaimā dīenuē herikkusu mage no sūgaku moderu}
\item Course 2 Lecture 25-27 Article: \href{}{Sample: A Mathematical Model of the Regulation of Intracellular Signaling Cascades 細胞内シグナル伝達カスケードの調節の数学モデル Saibō-nai shigunaru dentatsu kasukēdo no chōsetsu no sūgaku moderu}
\item Course 2 Lecture 28-30 Article: \href{}{Sample: A Mathematical Model of Motif Mediation in the Heterotrimeric G-protein Signaling Pathway ヘテロ三量体Gタンパク質シグナル伝達経路におけるモチーフ調停の数学モデル Hetero san ryōtai G tanpakushitsu shigunaru dentatsu keiro ni okeru mochīfu chōtei no sūgaku moderu}
\end{enumerate}

%--------------------------------------------------------------------------------------%
%-------------------------Course 3---------------------------------------------------%
%--------------------------------------------------------------------------------------%
\section{\centerline{\textcolor{c4}{\tiny Course 3: Molecular Machine Learning with Topological Dynamics for Compound Discovery in Organelles}}}
\begin{enumerate}
\item Course 3 Lecture 1-3 Article: \href{}{Sample: A Mathematical Model of Ribosome Flow リボソーム流動の数学モデル Ribosōmu ryūdō no sūgaku moderu}
\item Course 3 Lecture 4-6 Article: \href{}{Sample: A Mathematical Model of Transcriptional, Translational, Protein Folding, and Post Translational Errors 転写、翻訳、タンパク質折りたたみ、および翻訳後エラーの数学モデル Tensha, hon'yaku, tanpakushitsu oritatami, oyobi hon'yaku-go erā no sūgaku moderu}
\item Course 3 Lecture 7-9 Article: \href{}{Sample: A Mathematical Model of the Biosynthesis of Alkaloids from Shikimate Pathway シキミ酸経路からのアルカロイドの生合成の数学モデル Shikimi san keiro kara no arukaroido no nama gōsei no sūgaku moderu}
\item Course 3 Lecture 10-12 Article: \href{}{Sample: A Mathematical Model of WDR and G Quadruplexes WDRおよびG四重鎖の数学モデル WDR oyobi G shi kasanegusari no sūgaku moderu}
\item Course 3 Lecture 13-15 Article: \href{}{Sample: A QSAR Feature Matrix Design For Protein-Compound Interaction タンパク質と化合物の相互作用のためのQSAR機能マトリックス設計 Tanpakushitsu to kagōbutsu no sōgo sayō no tame no QSAR kinō matorikkusu sekkei}
\item Course 3 Lecture 16-18 Article: \href{}{Sample: A Mathematical Model of Ribosome Model with Circular mRNA 円形mRNAを用いたリボソームモデルの数学的モデル Enkei mRNA o mochiita ribosōmumoderu no sūgaku-teki moderu}
\item Course 3 Lecture 19-21 Article: \href{}{Sample: A Mathematical Model of Stress Induced Upregulation in Protein Responses ストレスの数学的モデルは、タンパク質応答のアップレギュレーションを誘発しました Sutoresu no sūgaku-teki moderu wa, tanpakushitsu ōtō no appuregyurēshon o yūhatsu shimashita}
\item Course 3 Lecture 22-24 Article: \href{}{Sample: A Mathematical Model of the Assembly of a 20S Proteasome with Peptide Post-Translational Modification ペプチド翻訳後修飾を伴う20Sプロテアソームのアセンブリの数学モデル Pepuchido hon'yakugoshūshoku o tomonau 20 esupuro teasōmu no asenburi no sūgaku moderu}
\item Course 3 Lecture 25-27 Article: \href{}{Sample: A Molecular Machine Learning Algorithm of Multi-Protein Complexes in The Digestion System 消化システムにおけるマルチタンパク質複合体の分子機械学習アルゴリズム Shōka shisutemu ni okeru maruchi tanpakushitsu fukugō-tai no bunshi kikai gakushū arugorizumu}
\item Course 3 Lecture 28-30 Article: \href{}{Sample: A Mathematical Model of Normal Mode Superposition in Compound-Protein Interaction 化合物間相互作用におけるノーマルモード重ね合わせの数学モデル Kagōbutsu-kan sōgo sayō ni okeru nōmarumōdo kasane-awase no sūgaku moderu}
\end{enumerate}
%--------------------------------------------------------------------------------------%
%----------------------------Course 4------------------------------------------------%
%--------------------------------------------------------------------------------------%
\section{\centerline{\textcolor{c4}{\tiny Course 4: Music and Mathematics Classical Music and Smooth Jazz Modeling, Analysis and Simulation}}}
\begin{enumerate}
\item \href{}{Course 4 Lecture 1-3 Sample: A Delayed Markov Model of Dance Choreography based on Topological Transformations 位相変換に基づくダンス振り付けの遅延マルコフモデル Isō henkan ni motodzuku dansu furitsuke no chien marukofumoderu}
\item \href{}{Course 4 Lecture 4-6 Sample: A Mathematical Model for Romantic Music Based on Filter Design}
\item \href{}{Course 4 Lecture 7-9 Sample: A Mathematical Model for a Classical Music Composition クラシック音楽の作曲のための数学モデル Kurashikku ongaku no sakkyoku no tame no sūgaku moderu}
\item \href{}{Course 4 Lecture 10-12 Sample: A Mathematical Model for a Jazz Music Composition  ジャズ音楽の作曲のための数学モデル Jazu ongaku no sakkyoku no tame no sūgaku moderu}
\item \href{}{Course 4 Lecture 13-15 Sample: Semi-Markov Representation for Octave Transition Model with Non-Symmetric Jumps in the Transition Rate Designs
遷移速度設計における非対称ジャンプを伴うオクターブ遷移モデルのセミマルコフ表現 Sen'i sokudo sekkei ni okeru hitaishō janpu o tomonau okutābu sen'i moderu no semimarukofu hyōgen}
\item \href{}{Course 4 Lecture 16-18 Sample: Frequency Based Markov Chains with Recurrent Distributional Positive Feedback Models for Classical Music Compositions
クラシック音楽作曲のための反復分布正帰還モデルを用いた周波数ベースのマルコフ連鎖 Kurashikku ongaku sakkyoku no tame no hanpuku bunpu sei kikan moderu o mochiita shūhasū bēsu no marukofu rensa}
\item \href{}{Course 4 Lecture 19-21 Sample: Recommender System Design of Beginner Piano Classical Composition Motifs Based on Sound Complexity and Diversity Indices 音の複雑さと多様性の指標に基づく初心者のピアノ古典作曲モチーフの推薦システム設計 Oto no fukuzatsu-sa to tayō-sei no shihyō ni motodzuku shoshinsha no piano koten sakkyoku mochīfu no suisen shisutemusetsukei}
\item \href{}{Course 4 Lecture 22-24 Sample: A Voice Response ChattoBotto for Numerical Sequence Cartography}
\item \href{}{Course 4 Lecture 25-27 Sample: A Mathematical Model for a Jazz Music Composition  ジャズ音楽の作曲のための数学モデル Jazu ongaku no sakkyoku no tame no sūgaku moderu}
\item \href{}{Course 4 Lecture 28-30 Sample: Motif Complexity Design of Modal Patterns for Smooth Jazz Compositions in Classical Duplet-Triplet Clusters モチーフの複雑さの設計古典的なデュプレット-トリプレットクラスターでのスムーズなジャズ曲のモーダルパターン Mochīfu no fukuzatsu-sa no sekkei koten-tekina de~yupuretto - toripurettokurasutā de no sumūzuna jazu kyoku no mōdarupatān}
\end{enumerate}
%--------------------------------------------------------------------------------------%
%------------------Music Research Compositions Section-------------------------%
%--------------------------------------------------------------------------------------%
\section{\centerline{\textcolor{c4}{\tiny Music Research Compositions}}}
\tiny
\begin{enumerate}
\item \href{https://github.com/MathematicalLearningSpace/Music-Room-A/blob/master/Composition_1.xml}{Music Composition 1}
\item \href{https://github.com/MathematicalLearningSpace/Music-Room-A/blob/master/Composition_2.xml}{Music Composition 2}
\item \href{https://github.com/MathematicalLearningSpace/Music-Room-A/blob/master/Composition_3.xml}{Music Composition 3}
\item \href{https://github.com/MathematicalLearningSpace/Music-Room-A/blob/master/Composition_4.xml}{Music Composition 4}
\item \href{https://github.com/MathematicalLearningSpace/Music-Room-A/blob/master/Composition_5.xml}{Music Composition 5}
\item \href{https://github.com/MathematicalLearningSpace/Music-Room-A/blob/master/Composition_6.xml}{Music Composition 6}
\item \href{https://github.com/MathematicalLearningSpace/Music-Room-A/blob/master/Composition_7.xml}{Music Composition 7}
\item \href{https://github.com/MathematicalLearningSpace/Music-Room-A/blob/master/Composition_8.xml}{Music Composition 8}
\item \href{https://github.com/MathematicalLearningSpace/Music-Room-A/blob/master/Composition_9.xml}{Music Composition 9}
\item \href{https://github.com/MathematicalLearningSpace/Music-Room-A/blob/master/Composition_10.xml}{Music Composition 10}
\end{enumerate}
\newpage
%---------------------------------------------------------------------------------------%
%-------------------------------------------Awards Section--------------------------%
%---------------------------------------------------------------------------------------%
\section{\centerline{\textcolor{c4}{\tiny Awards and Scholarships}}}
\tiny
\begin{enumerate} \itemsep -2pt
\item Full tuition research assistantship, Regional Research Institute 1986  \\
\item Qualifying PhD Exam in Econometrics-Pass with Distinction\\
\item Research Quality Award - Edinboro University of PA 1990\\
\item Wall Street Journal Achievement Award 1985\\
\item Burns Scholarship for Outstanding Achievement in Social Sciences 1985\\
\item Presidential Scholar 1985,1986\\
\end{enumerate}

%---------------------------------------------------------------------------------------%
%-------------------------------------------Languages-------------------------------%
%---------------------------------------------------------------------------------------%
\begin{enumerate}
\item \href{}{\textcolor{c4}{Beginner Second Language Acquisition Studies: English - Japanese Edition (初級第二言語習得研究:英語-日本語版)}}
\end{enumerate}
%------------------------------------------------------------------------------------------%
%-------------------------------------------Hobbies Section-----------------------------%
%------------------------------------------------------------------------------------------%
\section{\centerline{\textcolor{c4}{\tiny Hobbies}}}
\tiny
\begin{enumerate}
\item Composing Classical and Smooth Jazz music
\item Second Language Acquistion with Mandarin and Japanese
\item Drawing, Graphic Design and Botanical Scientific Illustrations
\item Playing the Piano and Digital Keyboard 
\item Sports-Basketball and Table and Court Tennis
\item International Music
\item Fashion and Clothing Design
\end{enumerate}

\end{CJK}


\end{document}
