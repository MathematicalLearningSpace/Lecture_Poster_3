\documentclass{recipecard}
\usepackage{listings}

\begin{document}
\vspace{8pt}

\ingredient{1 tblspoon of salt}
\ingredient{2 tblspoons of sugar}
\ingredient{1 cup of Sushi Rice}
\ingredient{2 tblspoons of Basil}
\ingredient{2 radishes}
\ingredient{2 sheets of Nori}
\ingredient{2 tblspoons of Basil}
\ingredient{1.25 ratio of water to rice}
\ingredient{2.5 tblspoons of rice vinegar}
\ingredient{2 tblspoons of Lemon Pepper}
\ingredient{1 cup of carrots}
\ingredient{1.5 tablespoon of grapeseed oil}
\ingredient{3 tblspoons of Kum chun Soy Sauce}

\begin{recipe}{Norimaki: Brassicaceae or Radish Roll}{}

 \textbf{(1)} Wash Rice 
 \textbf{(2)} Add water, Basil, grapeseed oil, and salt. Boil mixture at temperature in the interval [70,120] Celsius. Elevation (i.e. atmosphere pressure) and Salt changes the boiling point. 
 \textbf{(3)} Mix water and rice. Reduce heat to low for 15 minutes. 
 \textbf{(4)} Slice radish and carrots into desired shapes. 
 \textbf{(5)} Add rice vinegar and sugar to rice. 
 \textbf{(6)} Make a thin layer of uniform distributed rice on each sheet of red algae genus Pyropia. Either P. yezoensis or P.tenera 
 \textbf{(7)} Add a horizontal line graph of carrots. 
 \textbf{(8)} Add a horizontal line graph of radish. 
 \textbf{(9)} Uniformly distribute the Lemon pepper on each layer.
 \textbf{(10)} Roll each sheet with shiny side down on handmade Bamboo Rolling Mat. 
 \textbf{(11)} Slice each cylinder into the number of sections.
 \textbf{(12)} Serve the two Norimaki with presentation designs.
\end{recipe}

\end{document}